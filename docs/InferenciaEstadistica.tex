% Options for packages loaded elsewhere
\PassOptionsToPackage{unicode}{hyperref}
\PassOptionsToPackage{hyphens}{url}
%
\documentclass[
]{book}
\usepackage{lmodern}
\usepackage{amssymb,amsmath}
\usepackage{ifxetex,ifluatex}
\ifnum 0\ifxetex 1\fi\ifluatex 1\fi=0 % if pdftex
  \usepackage[T1]{fontenc}
  \usepackage[utf8]{inputenc}
  \usepackage{textcomp} % provide euro and other symbols
\else % if luatex or xetex
  \usepackage{unicode-math}
  \defaultfontfeatures{Scale=MatchLowercase}
  \defaultfontfeatures[\rmfamily]{Ligatures=TeX,Scale=1}
\fi
% Use upquote if available, for straight quotes in verbatim environments
\IfFileExists{upquote.sty}{\usepackage{upquote}}{}
\IfFileExists{microtype.sty}{% use microtype if available
  \usepackage[]{microtype}
  \UseMicrotypeSet[protrusion]{basicmath} % disable protrusion for tt fonts
}{}
\makeatletter
\@ifundefined{KOMAClassName}{% if non-KOMA class
  \IfFileExists{parskip.sty}{%
    \usepackage{parskip}
  }{% else
    \setlength{\parindent}{0pt}
    \setlength{\parskip}{6pt plus 2pt minus 1pt}}
}{% if KOMA class
  \KOMAoptions{parskip=half}}
\makeatother
\usepackage{xcolor}
\IfFileExists{xurl.sty}{\usepackage{xurl}}{} % add URL line breaks if available
\IfFileExists{bookmark.sty}{\usepackage{bookmark}}{\usepackage{hyperref}}
\hypersetup{
  pdftitle={Inferencia},
  pdfauthor={Rodrigo Zepeda-Tello y Luis Carlos Bernal},
  hidelinks,
  pdfcreator={LaTeX via pandoc}}
\urlstyle{same} % disable monospaced font for URLs
\usepackage{longtable,booktabs}
% Correct order of tables after \paragraph or \subparagraph
\usepackage{etoolbox}
\makeatletter
\patchcmd\longtable{\par}{\if@noskipsec\mbox{}\fi\par}{}{}
\makeatother
% Allow footnotes in longtable head/foot
\IfFileExists{footnotehyper.sty}{\usepackage{footnotehyper}}{\usepackage{footnote}}
\makesavenoteenv{longtable}
\usepackage{graphicx,grffile}
\makeatletter
\def\maxwidth{\ifdim\Gin@nat@width>\linewidth\linewidth\else\Gin@nat@width\fi}
\def\maxheight{\ifdim\Gin@nat@height>\textheight\textheight\else\Gin@nat@height\fi}
\makeatother
% Scale images if necessary, so that they will not overflow the page
% margins by default, and it is still possible to overwrite the defaults
% using explicit options in \includegraphics[width, height, ...]{}
\setkeys{Gin}{width=\maxwidth,height=\maxheight,keepaspectratio}
% Set default figure placement to htbp
\makeatletter
\def\fps@figure{htbp}
\makeatother
\setlength{\emergencystretch}{3em} % prevent overfull lines
\providecommand{\tightlist}{%
  \setlength{\itemsep}{0pt}\setlength{\parskip}{0pt}}
\setcounter{secnumdepth}{5}
\usepackage{booktabs}
\usepackage[framemethod=TikZ]{mdframed}
\usepackage{xcolor}
\definecolor{ejemplocolor}{HTML}{0A7000}
\definecolor{recuadrocolor}{HTML}{3DB295}
\definecolor{teoremacolor}{HTML}{900C3F}
\definecolor{propiedadescolor}{HTML}{C301AC}
\definecolor{importantecolor}{HTML}{FF0000}
\definecolor{corolariocolor}{HTML}{0CABAD}
\definecolor{lemacolor}{HTML}{0BA50B}
\definecolor{ejerciciocolor}{HTML}{183283}
\definecolor{definicioncolor}{HTML}{183283}
\definecolor{formulacolor}{HTML}{137AA4}


\newenvironment{Ejemplo}
{\begin{mdframed}[
  linecolor=ejemplocolor,
  skipabove=12pt,
  skipbelow=12pt,
  roundcorner=20pt,
  splittopskip=2\topsep]}
{\end{mdframed}}

\newenvironment{Definicion}
{\begin{mdframed}[
  linecolor=definicioncolor,
  skipabove=12pt,
  skipbelow=12pt,
  roundcorner=20pt,
  splittopskip=2\topsep]}
{\end{mdframed}}

\newenvironment{Recuadro}
{\begin{mdframed}[
  linecolor=recuadrocolor,
  skipabove=12pt,
  skipbelow=12pt,
  roundcorner=20pt,
  splittopskip=2\topsep]}
{\end{mdframed}}

\newenvironment{Teorema}
{\begin{mdframed}[
  linecolor=teoremacolor,
  skipabove=12pt,
  skipbelow=12pt,
  roundcorner=20pt,
  splittopskip=2\topsep]}
{\end{mdframed}}

\newenvironment{Propiedades}
{\begin{mdframed}[
  linecolor=propiedadescolor,
  skipabove=12pt,
  skipbelow=12pt,
  roundcorner=20pt,
  splittopskip=2\topsep]}
{\end{mdframed}}

\newenvironment{Importante}
{\begin{mdframed}[
  linecolor=importantecolor,
  skipabove=12pt,
  skipbelow=12pt,
  roundcorner=20pt,
  splittopskip=2\topsep]}
{\end{mdframed}}

\newenvironment{Corolario}
{\begin{mdframed}[
  linecolor=corolariocolor,
  skipabove=12pt,
  skipbelow=12pt,
  roundcorner=20pt,
  splittopskip=2\topsep]}
{\end{mdframed}}

\newenvironment{Lema}
{\begin{mdframed}[
  linecolor=lemacolor,
  skipabove=12pt,
  skipbelow=12pt,
  roundcorner=20pt,
  splittopskip=2\topsep]}
{\end{mdframed}}

\newenvironment{Ejercicio}
{\begin{mdframed}[
  linecolor=ejerciciocolor,
  skipabove=12pt,
  skipbelow=12pt,
  roundcorner=20pt,
  splittopskip=2\topsep]}
{\end{mdframed}}

\newenvironment{Formula}
{\begin{mdframed}[
  linecolor=formulacolor,
  skipabove=12pt,
  skipbelow=12pt,
  roundcorner=20pt,
  splittopskip=2\topsep]}
{\end{mdframed}}
\usepackage[]{natbib}
\bibliographystyle{apalike}

\title{Inferencia}
\author{Rodrigo Zepeda-Tello y Luis Carlos Bernal}
\date{2021-01-12}

\begin{document}
\maketitle

{
\setcounter{tocdepth}{1}
\tableofcontents
}
\hypertarget{intro}{%
\chapter{Introducción}\label{intro}}

\hypertarget{estaduxedstica-y-muestras}{%
\section{Estadística y muestras}\label{estaduxedstica-y-muestras}}

La \href{https://plato.stanford.edu/entries/statistics/\#StaInd}{enciclopedia Stanford de filosofía} establece la siguiente definición de estadística\footnote{Traducción y subrayado de Rodrigo}.

\begin{Definicion}
\textbf{Estadística} La estadística es una disciplina matemática y
conceptual que se enfoca en la relación entre datos e hipótesis. Los
datos son registros de observaciones o eventos en un estudio científico,
por ejemplo, un conjunto de mediciones de individuos de una población.
Los datos que son obtenidos se conoce como la muestra, datos muestrales,
o simplemente los datos, y todas las posibles muestras posibles en un
estudio forman una colección llamada el espacio muestra. Las hipótesis,
por su parte, son enunciados generales sobre el sistema objetivo de la
investigación científica, por ejemplo, expresar un hecho general sobre
todos los individuos en la población. Una hipótesis estadística es un
enunciado general que puede ser expresada como una distribución de
probabilidad sobre el espacio muestral, es decir, ésta determina una
probabilidad para cada una de las posibles muestras.
\end{Definicion}

De manera breve, la estadística es una disciplina que se encarga de, a través de muestras (cuantificadas como datos), describir el mundo. Y hay muchas cosas por describir: asociaciones, causalidad, realizar predicciones, establecer mecanismos de funcionamiento de objetos, etc. Es así como se establece su objetivo el cual de acuerdo con \citet{wackerly} es:

\begin{quote}
realizar una inferencia sobre la población con base en la información contenida en una muestra de dicha población y proveer una medida asociada de qué tan buena es la inferencia.
\end{quote}

Dentro de la definición previa y objetivos hay que destacar varios términos que son de importancia. La primera es la \textbf{población}, cualquier conjunto (no vacío) de objetos. Una \textbf{población} es lo más general posible, no necesariamente involucra personas o seres vivos. Algunos ejemplos de poblaciones incluyen: las personas que viven en Guatemala (si me interesa saber algo de los guatemaltecos en general), los árboles del Amazonas (si quiero saber cosas de ecología), los perros callejeros en Ciudad de México, los consumidores de una marca de cereal, los coches que transitan por Dubai, los granos de arena en una playa específica de Cancún, las células T dentro de los seres humanos o los metales pesados.

Más relevante que la población (para nuestros propósitos) es la \textbf{población objetivo} El conjunto de elementos que formarán parte del estudio. Definir la \textbf{población objetivo} es complicado en algunas situaciones; por ejemplo, si se desea saber si \emph{los mexicanos} están a favor o en contra de legalizar la marihuana hay que establecer quiénes son \emph{los mexicanos}. ¿Cuentan las personas con nacionalidad mexicana que residen en el extranjero? ¿Cuentan los menores de edad? ¿Qué pasa con los extranjeros que son residentes? De nuevo, la población objetivo no necesariamente son personas, es sólo aquello que nos interesa medir.

Idealmente el estudio estadístico sería sobre la población objetivo. Por ejemplo, si nos interesa estudiar la evolución de los enfermos de VIH, la \textbf{población objetivo} serían los enfermos. Sin embargo, en el mundo real es imposible conseguir a toda la población objetivo (dentro de los enfermos, por ejemplo, están aquellos que aún no saben que tienen la enfermedad y no acudirían a nuestro estudio). La \textbf{población muestreada} resulta de esta dificultad. La \textbf{población muestreada} es el conjunto de elementos sobre los cuales se construyó la muestra para el análisis estadístico. En el caso de los enfermos de VIH la \textbf{población muestreada} podrían ser las personas que para una fecha específica habían sido diagnosticadas (y nos olvidamos de quienes desconocen su diagnóstico) o toda la población mexicana (y llevamos kits de diagnóstico con nosotros cuando diagnostiquemos). En encuestas de consumo, por ejemplo, usualmente no se muestrean zonas remotas o de muy bajos recursos por lo que la \textbf{población muestreada} no coincide con la \textbf{población objetivo} (todos los consumidores) sino que son sólo los consumidores de mayor poder adquisitivo. En encuestas de elecciones si bien la población objetivo son \emph{todas las personas que voten el día de la elección}, como la mayoría se hacen \emph{antes} de la elección (exceptuando las de salida) entonces se aproxima la definición de \emph{votante} buscando incluir sólo aquellos que estén registrados en el padrón electoral o bien aquellos que al ser encuestados digan que \emph{sí} van a votar. Aquí la \textbf{población muestreada} tampoco coincide con la objetivo.

Una \textbf{muestra} es un subconjunto de la población muestreada. Si la muestra coincide con la población muestreada (es decir, muestreaste a todo el mundo) se dice que es un \textbf{censo}. Si se tiene un censo se conoce TODA la población por lo que no es necesario hacer ningún análisis de inferencia (ya sabes todo de todos). Puedes realizar predicciones o descripciones. Ejemplos de censos son las encuestas de fin de cursos, las calificaciones de todo un grupo o el registro de todas las compras de todas las personas en una tienda en línea.

\begin{quote}
\textbf{Ojo} No hay que confundir la definición de \textbf{muestra} con la definición estadística de \textbf{muestra aleatoria} (ver más adelante) la cual es un tipo muy específico de muestra obtenida bajo reglas restrictivas.
\end{quote}

Finalmente hay que definir \textbf{inferencia}, el propósito de estas notas. Para ello usaremos el ejemplo y una versión adaptada de la definición de \citet{boghossian2014inference}. Considera que sabes dos verdades:

\begin{enumerate}
\def\labelenumi{\arabic{enumi}.}
\item
  Llovió anoche
\item
  Cuando llueve el suelo se moja
\end{enumerate}

por lo que esta mañana \emph{infieres} que el suelo estará mojado y sales de tu casa con botas y no con chanclas. El proceso de \emph{inferir} parece una consecuencia lógica de las premisas 1 y 2 pero no lo es exactamente: hoy es otro día y si hizo suficiente calor en la noche el agua pudo haberse evaporado del suelo. De ahí que definamos inferencia como:

\begin{quote}
Realizar un juicio el cual se explica a partir de premisas que suponemos verdaderas.
\end{quote}

En particular \textbf{la inferencia estadística} será la rama de la estadística cuyo propósito es

\begin{quote}
Realizar juicios probabilísticos a partir de datos que suponemos verdaderos.
\end{quote}

Aquí es necesario desglosar un poco la definición:

\begin{itemize}
\item
  Se habla de \textbf{juicios probabilísticos} pues nuestros juicios nunca van a ser tan certeros como \texttt{el\ suelo\ està\ mojado}. Más bien van a ser del estilo \texttt{hay\ una\ probabilidad\ muy\ alta\ de\ que\ el\ suelo\ esté\ mojado} o \texttt{nueve\ de\ cada\ diez\ veces\ el\ suelo\ estará\ mojado}.
\item
  La \textbf{suposición de verdad} de los datos es muy relevante. Imagina el siguiente experimento: tu amiga borracha durante una fiesta se le ocurre que, de la nada, desarrolló poderes de psíquica y puede adivinar el futuro resultado de una moneda (cara o cruz). Tiras una moneda diez veces y todas las veces tu amiga hace una predicción correcta. \emph{Considerando los datos como verdad concluirías que tu amiga es psíquica}. Una observación a profundidad de la moneda quizá te revele que es una moneda truqueada que siempre cae en cara. En ese momento cambiarías la \textbf{suposición de verdad} de los datos y la inferencia de que tu amiga es psíquica.
\end{itemize}

A lo largo de este libro aprenderemos lo básico para realizar inferencias estadísticas: observar datos y suponer verdades a partir de ellos. Tristemente la estadística nunca nos va a poder dar la verdad absoluta pero, si lo hacemos bien, es quizá lo más cerca que podamos estar de ella.

\hypertarget{modelos}{%
\section{Modelos}\label{modelos}}

La estadística funciona a partir de la construcción de \textbf{modelos}. Estos pretenden ser una forma de describir el mundo mediante teoría de la probabilidad y lo que se busca es utilizar dicha teoría para realizar inferencias. Estos modelos teóricos representan la forma en la que suponemos funciona la población. Para propósitos de estas notas diremos que los modelos viven \emph{en el mundo de los modelos} o \emph{mundo de las ideas}. Los datos observados, para poder distinguirlos, viven en \emph{el mundo real}. Muchos de los modelos (no todos) se componen de \textbf{parámetros} que requieren para poder funcionar los cuales son estimados mediante \textbf{estadísticos} que se construyen a partir de los datos.

Para nuestros propósitos, los modelos que usaremos siempre construirán una población de la siguiente forma:

\begin{Definicion}
\hypertarget{poblaciuxf3n}{%
\subsubsection{Población}\label{poblaciuxf3n}}

Una población es un conjunto no vacío de variables (o vectores)
aleatorias. \[
\mathcal{X} = \{ X_1, X_2, \dots \}
\]
\end{Definicion}

Una población no necesariamente es finita. Por ejemplo, si nos interesa saber el tiempo que tarda un cliente en ser atendido en una llamada telefónica al banco quizá podemos suponer que la llamada telefónica tiene una duración descrita por un modelo \(\text{Exponencial}(\lambda)\). La población sería el conjunto infinito de todas las posibles llamadas telefónicas que se pueden realizar bajo este modelo. Por otro lado, un ejemplo finito de una población, son las caras de una moneda en un experimento donde busquemos, para una moneda específica, si caen más caras que cruces (cae más de un lado que del otro).

Una \textbf{muestra} es cualquier subconjunto (posiblemente infinito también) de la población.

\begin{Definicion}
\hypertarget{muestra}{%
\subsubsection{Muestra}\label{muestra}}

Una muestra \(\mathcal{M}\) de una población \(\mathcal{X}\) es
cualquier subconjunto no vacío de \(\mathcal{X}\). Es decir,
\(\mathcal{M}\) es una muestra de \(\mathcal{X}\) si: \[
 \mathcal{M} \subseteq \mathcal{X}
\]
\end{Definicion}

Pocas veces hablaremos de \emph{muestras} de manera general y nos enfocaremos, sobre todo, en \textbf{muestras aleatorias}:

\begin{Definicion}
\hypertarget{muestra-aleatoria}{%
\subsubsection{Muestra aleatoria}\label{muestra-aleatoria}}

Una muestra aleatoria de tamaño \(n\), \(\mathcal{X}_{(n)}\), de una
población \(\mathcal{X}\) es un subconjunto finito (de tamaño \(n\)), no
vacío de \(\mathcal{X}\) donde sus elementos son \textbf{variables
aleatorias independientes idénticamente distribuidas}. Es decir,
\(\mathcal{X}_{(n)}\) es una muestra de \(\mathcal{X}\) si:

\begin{enumerate}
\def\labelenumi{\arabic{enumi}.}
\item
  \textbf{Es una muestra}: \(\mathcal{X}_{(n)} \subseteq \mathcal{X}\),
\item
  \textbf{de tamaño \(n\)}:
  \(\textrm{Cardinalidad}\Big( \mathcal{X}_{(n)} \Big) = n\),
\item
  \textbf{con variables independientes}: si
  \(X_i, X_j \in \mathcal{X}_{(n)}\) entonces
  \(\mathbb{P}(X_i \in A , X_j \in B) = \mathbb{P}(X_i \in A)\cdot\mathbb{P}(X_j \in B)\)
  para \(A,B\) conjuntos \emph{medibles},
\item
  \textbf{idénticamente distribuidas}: para todo \(i = 1,2,\dots, n\) se
  tiene que \(X_i\) tiene función de distribución acumulada \(F_X\).
\end{enumerate}
\end{Definicion}

El punto 4. de la definición pide que todas las variables descritas tengan la misma distribución. Por ejemplo, podemos pedir que todas sean exponenciales con el mismo parámetro o todas sean gamma con los mismos parámetros. El punto es que todas las variables aleatorias estén descritas con el mismo modelo y sean independientes entre sí.

El punto 3. de la definición puede escribirse de otras formas más amigables, por ejemplo, si suponemos que las variables aleatorias son continuas y tienen densidad \(f_X\) entonces la independencia puede escribirse como:
\[
f_X(x_i, x_j) = f_X(x_i) \cdot f_X(x_j)
\]
mientras que si son discretas con función de masa de probabilidad \(p_X\) tenemos:
\[
p_X(x_i, x_j) = p_X(x_i) \cdot p_X(x_j)
\]

La \textbf{muestra observada} así como \textbf{la muestra aleatoria observada} es el conjunto de \emph{datos} que realmente viste. Mientras que la \textbf{muestra} y la \textbf{muestra aleatoria} viven \emph{en el mundo de los modelos} y son variables aleatorias (constructos teóricos, como sabes, bastante complejos), la \textbf{muestra observada} es lo que se midió. Antes de dar la definición veamos un ejemplo con un dado.

\begin{Ejemplo}
\hypertarget{tiro-de-un-dado}{%
\subsubsection{Tiro de un dado}\label{tiro-de-un-dado}}

Se realiza un experimento para saber si un dado es justo (todos los
lados tienen la misma probabilidad). Para ello se tira el dado
\(n = 10\) veces y se registran los tiros: \(2,6,1,3,3,3,5,1,3,2\).

\emph{Mundo del modelo}

La población en este caso es el conjunto infinito de todos los posibles
tiros del dado. De ese conjunto obtenemos una muestra aleatoria de
tamaño \(n = 10\) (suponemos que los tiros son independientes entre sí)
dada por: \[
X_{(n)} = \{ X_1, X_2, X_3, \dots, X_{10}\}
\] donde \(X_i\) tiene la siguiente distribución: \[
\mathbb{P}(X_i = z) = 
\begin{cases}
p_1 & \text{ si } z = 1 \\
p_2 & \text{ si } z = 2 \\
p_3 & \text{ si } z = 3 \\
p_4 & \text{ si } z = 4 \\
p_5 & \text{ si } z = 5 \\
p_6 & \text{ si } z = 6 \\
0 & \text{ en otro caso.}
\end{cases}
\] donde \(\sum_{k = 1}^n p_k = 1\) y \(p_{k} \geq 0\) para todo \(k\).
Lo que interesa en este estudio es \emph{inferir} quiénes son las
\(p_k\) para determinar si es más probable que caiga en un lado que en
otro. Las \(p_k\) se conocen como parámetros.

\emph{Mundo real}

Ya en la realidad en esos \(10\) tiros no observamos cualquier cosa,
observamos valores específicos que hacen que \textbf{la muestra
aleatoria observada} sea: \[
s_n = \{x_1, x_2, \dots, x_10 \} = \{2,6,1,3,3,3,5,1,3,2\}.
\] Por supuesto que repitiendo el experimento (volviendo a tirar el dado
10 veces) lo más probable es que la \textbf{muestra aleatoria observada}
cambie (y veamos otros números) pero el modelo, reflejado en la
\textbf{muestra aleatoria} (teórica), permanezca inmutable. Una forma de
estimar las probabilidades podría ser mediante proporciones y calcular,
por ejemplo, la probabilidad de que aparezca \(1\) como: \[
\hat{p}_1 = \frac{\text{Veces que aparece 1}}{n} = \frac{2}{10}
\] En este caso, \(\hat{p}_1\) dado por \(\frac{2}{10}\) es un
\textbf{estimador observado} de la verdadera probabilidad \(p_1\) que
vive en el mundo de los modelos (y jamás podremos conocer)
\end{Ejemplo}

Armados con el ejemplo anterior realicemos la definición de las muestras observadas:

\begin{Definicion}
\hypertarget{muestra-observada}{%
\subsubsection{Muestra observada}\label{muestra-observada}}

Una muestra observada es una colección no vacía de valores codificados
como números reales los cuales corresponden a realizaciones de una
muestra \(\mathcal{X}\). Usualmente la denotamos: \[
s = \{ x_1, x_2, \dots\}
\] donde las \(x_i\) \textbf{NO SON VARIABLES ALEATORIAS} sino que son
datos \textbf{fijos} ya observados.
\end{Definicion}

\begin{Definicion}
\hypertarget{muestra-aleatoria-observada}{%
\subsubsection{Muestra aleatoria
observada}\label{muestra-aleatoria-observada}}

Una muestra aleatoria observada es una colección no vacía de tamaño
\(n\) de valores codificados como números reales los cuales corresponden
a realizaciones de una muestra aleatoria \(\mathcal{X}_{(n)}\). En
particular suponemos que \(x_1\) es el valor observado de la variable
aleatoria \(X_1\), \(x_2\) es el valor observado de la variable
aleatoria \(X_2\) y así sucesivamente. Generalmente la denotamos por: \[
s_{(n)} = \{ x_1, x_2, \dots, x_n\}
\] donde las \(x_i\) \textbf{NO SON VARIABLES ALEATORIAS} sino que son
datos \textbf{fijos} ya observados.
\end{Definicion}

Veamos un segundo ejemplo:

\begin{Ejemplo}
\hypertarget{cantidad-de-personas-que-llegan-a-una-tienda}{%
\subsubsection{Cantidad de personas que llegan a una
tienda}\label{cantidad-de-personas-que-llegan-a-una-tienda}}

En muchos casos la llegada de personas se supone que sigue una
distribución Poisson. En este caso nos interesa estimar el número
promedio de personas por día que hay en una tienda de la cual se han
medido las siguientes cantidades (por día). Suponemos que las llegadas
son independientes entre sí (la cantidad de gente que llegó un día no
influye en la cantidad que llegó el otro).

\begin{longtable}[]{@{}ll@{}}
\toprule
\textbf{Día} & \textbf{Número de personas}\tabularnewline
\midrule
\endhead
1 & 50\tabularnewline
2 & 45\tabularnewline
3 & 60\tabularnewline
4 & 65\tabularnewline
5 & 55\tabularnewline
6 & 40\tabularnewline
\bottomrule
\end{longtable}

\emph{Mundo del modelo}

La población en este caso es el conjunto infinito de todas las posibles
formas en que en un día pueden llegar personas. De ese conjunto
obtenemos una muestra aleatoria de tamaño \(n = 6\) (suponemos que las
observaciones son independientes entre sí) dada por: \[
X_{(n)} = \{ X_1, X_2, X_3, X_4, X_5, X_6\}
\] donde las \(X_i \sim \text{Poisson}(\lambda)\) (todas con el mismo
\(\lambda\)). Recordamos que la media de una Poisson es \(\lambda\) por
lo que el \textbf{parámetro} que nos interesa estimar es \(\lambda\).

\emph{Mundo real}

A partir de las \(6\) llegadas observadas construimos \textbf{la muestra
aleatoria observada}: \[
s_n = \{x_1, x_2, x_3, x_4, x_5, x_6 \} = \{50, 45, 60, 65, 55, 40\}.
\] Una forma de estimar la media \(\lambda\) es mediante el siguiente
\textbf{estimador observado}: \[
\hat{\lambda} = \frac{1}{6} \sum_{i = 1}^6 x_i = 52.5
\] Ojo, esto no significa que \(\lambda\) \emph{sea} \(52.5\). Significa
que nuestra hipótesis de quién es \(\lambda\) es \(52.5\) y que
esperaríamos la próxima vez en la tienda \(52\) ó \(53\) personas. En el
mundo real \emph{quién sabe cuánto vale \(\lambda\)} , nuestra hipótesis
es que vale \(52.5\) pero eso no necesarimente es la realidad.
\end{Ejemplo}

Como ya establecimos, muchas veces el modelo utiliza un \textbf{parámetro} el cual es desconocido. A partir de los datos construimos un \textbf{estimador observado} el cual es nuestra hipótesis del verdadero valor del parámetro. En general va a ser imposible que le atinemos al \emph{verdadero} valor del parámetro pero la idea es que el \textbf{estimador observado} esté lo suficientemente cerca. En el ejemplo anterior nos gustaría, por ejemplo, que el verdadero parámetro quizá fuera \(\lambda = 52\) ó \(\lambda = 54\) pero nos sacaría mucho de onda que el parámetro real fuera \(\lambda = 1000000\).

\begin{Definicion}
\hypertarget{distirbuciuxf3n-paramuxe9trica}{%
\subsubsection{Distirbución
paramétrica}\label{distirbuciuxf3n-paramuxe9trica}}

Una función de distirbución acumulada es una \textbf{distribución
paramétrica} con parámetro \(\vec{\theta}\) si dada una colección de
distribuciones \[
\{ F_{\vec{\theta}} | \theta \in \Theta \}
\] determinar \(\vec{\theta}\) determina la distribución. Es decir, la
familia de distribuciones está indizada por \(\vec{\theta}\). A
\(\vec{\theta}\) se le conoce como el \textbf{parámetro} o
\textbf{vector de parámetros}.
\end{Definicion}

La definición anterior suena muy compleja sin embargo los ejemplos ya los conocemos.

\begin{Ejemplo}
\hypertarget{la-normal}{%
\subsubsection{La normal}\label{la-normal}}

La distribución normal es una distribución paramétrica con \[
\vec{\theta} = (\mu, \sigma^2)^T
\] el vector de parámetros dado por la media y la varianza.
\end{Ejemplo}

\begin{Ejemplo}
\hypertarget{la-exponencial}{%
\subsubsection{La exponencial}\label{la-exponencial}}

La distribución exponencial es una distribución paramétrica con \[
\theta = \lambda
\] el parámetro que establece la tasa de la exponencial.
\end{Ejemplo}

\begin{Ejemplo}
\hypertarget{la-normal-con-varianza-1}{%
\subsubsection{La normal con varianza
1}\label{la-normal-con-varianza-1}}

La distribución normal con varianza 1 es una distribución paramétrica
con \[
\theta = \mu
\] En este caso la varianza es conocida (\(\sigma^2 = 1\)) pero la media
no por eso sólo la media es el parámetro.
\end{Ejemplo}

Podemos entonces definir un \textbf{estimador}:

\begin{Definicion}
\hypertarget{estimador}{%
\subsubsection{Estimador}\label{estimador}}

Dada una distribución paramétrica \(F_{\theta}\) con parámetro
\(\theta\) un estimador \(\hat{\theta}\) de \(\theta\) es una variable
aleatoria que se construye como función de la muestra aleatoria: \[
\hat{\theta}: \mathcal{X}_{(n)} \to \Theta
\] Como \(\hat{\theta}\) es una función de la muestra aleatoria entonces
puede representarse como: \[
\hat{\theta} = \hat{\theta}(X_1, X_2, \dots, X_n)
\]
\end{Definicion}

Dado un conjunto de datos, el \textbf{estimador observado de \(\theta\)} es el estimador \(\hat{\theta}\) de \(\theta\) evaluado en los datos.

\begin{Definicion}
\hypertarget{estimador-observado}{%
\subsubsection{Estimador observado}\label{estimador-observado}}

Dada una distribución paramétrica \(F_{\theta}\) con parámetro
\(\theta\) con estimador \(\hat{\theta}\) y datos observados
\(s_{(n)} = \{x_1, x_2, \dots, x_n\}\) el \textbf{estimador observado}
corresponde a la evaluación de \(\hat{\theta}\) en \(s_{(n)}\); es
decir: \[
\hat{\theta}(x_1, x_2, \dots, x_n)
\]
\end{Definicion}

Veamos ejemplos para entender mejor cómo funciona esto.

\begin{Ejemplo}
\hypertarget{tiros-de-una-moneda}{%
\subsubsection{Tiros de una moneda}\label{tiros-de-una-moneda}}

Se tiene una moneda que cae más de un lado que del otro. Interesa
estimar \(p\) la probabilidad de que caiga cruz. Para ello se toma una
\textbf{muestra aleatoria} de \(5\) tiros de la moneda: \[
X_{(n)} = \{X_1, X_2, \dots, X_{5} \}
\] Suponemos que los tiros son independientes. El modelo entonces
implicaría que \[
X_i \sim \text{Bernoulli}(p)
\] para cada \(i = 1, 2, \dots, 5\). Si codificamos cruz como \(1\) y
cara como \(0\), la \textbf{muestra aleatoria observada} es: \[
s_{(n)} = \{1,1,1,0,1\} = \{x_1, x_2, \dots, x_5\}
\] donde tuvimos tres cruces continuas, luego una cara y finalmente una
cruz. Una opción de estimador observado sería contar la proporción de
cruces haciendo: \[
\hat{\theta}(x_1, \dots, x_5) = \frac{1}{n} \sum_{k = 1}^n x_i = \frac{4}{5}
\] de donde diríamos que nuestra hipótesis de cuánto vale el parámetro
\(p\) es \(4/5\). Por otro lado, el \textbf{estimador} teórico es: \[
\hat{\theta}(X_1, \dots, X_5) = \frac{1}{n} \sum_{k = 1}^n X_i 
\] el cual tiene una distribución de probabilidad sencilla pues
\(\sum_{k = 1}^n X_i \sim \textrm{Binomial}(n,p)\). Particularmente
podemos calcular su valor esperado, por ejemplo, \[
\mathbb{E}\Big[  \hat{\theta}(X_1, \dots, X_5) \Big]  =  \mathbb{E}\Big[  \frac{1}{n} \sum_{k = 1}^n X_i  \Big] = \frac{1}{n}\sum_{k = 1}^n \mathbb{E}\Big[X_i \Big] = \frac{1}{n}\sum_{k = 1}^n  p = \frac{1}{n} np = p
\] lo cual implica que el estimador, en promedio, devolvería el
parámetro que nos interesa (esta propiedad se conoce como \emph{ser
insesgado} y lo veremos más adelante).
\end{Ejemplo}

\begin{Ejemplo}
\hypertarget{error-de-mediciuxf3n-de-una-app}{%
\subsubsection{Error de medición de una
app}\label{error-de-mediciuxf3n-de-una-app}}

Una app que se dedica a medir la altura de edificios mediante la toma de
videos tiene un error de medición con distribución normal y cuya
varianza es \(1\). Interesa determinar el error de medición promedio, el
parámetro \(\mu\). Para ello se toman videos y se miden edificios para
obtener una colección de 7 errores de medición independientes en la
siguiente \textbf{muestra aleatoria}: \[
X_{(n)} = \{X_1, X_2, \dots, X_{5} \}
\] El modelo es \[
X_i \sim \text{Normal}(\mu, 1)
\] para cada \(i = 1, 2, \dots, 7\). Si los datos fueron:

\begin{longtable}[]{@{}ll@{}}
\toprule
\textbf{Edificio} & \textbf{Error de medición}\tabularnewline
\midrule
\endhead
Bellas Artes & 12.11\tabularnewline
Torre Latinoamericana & 40.54\tabularnewline
Catedral Metropolitana & 22.07\tabularnewline
Palacio Nacional & 15.22\tabularnewline
Rectoría de la UNAM & 45.18\tabularnewline
Guerrero Chimalli & 33.39\tabularnewline
Estadio Azteca & 41.76\tabularnewline
\bottomrule
\end{longtable}

la \textbf{muestra aleatoria observada} en este caso correspondió : \[
s_{(n)} = \{12.11,40.54,22.07,15.22,45.18, 33.39, 41.76\} = \{x_1, x_2, \dots, x_7\}
\] Una opción de estimador observado sería calcular la media muestral
haciendo: \[
\hat{\theta}(x_1, \dots, x_7) = \frac{1}{n} \sum_{k = 1}^n x_i = 30.03857
\] de donde diríamos que nuestra hipótesis de cuánto vale el parámetro
\(\mu\) es \(30.03857\). Por otro lado, el \textbf{estimador} teórico
es: \[
\hat{\theta}(X_1, \dots, X_7) = \frac{1}{n} \sum_{k = 1}^n X_i 
\] tiene una distribución de probabilidad sencilla pues sabemos que
\(\sum_{k = 1}^n X_i \sim \textrm{Normal}(\mu,\sigma^2)\).
Particularmente podemos calcular su valor esperado, por ejemplo, \[
\mathbb{E}\Big[  \hat{\theta}(X_1, \dots, X_7) \Big]  =  \mathbb{E}\Big[  \frac{1}{n} \sum_{k = 1}^n X_i  \Big] = \frac{1}{n}\sum_{k = 1}^n \mathbb{E}\Big[X_i \Big] = \frac{1}{n}\sum_{k = 1}^n  \mu = \frac{1}{n} n\mu = \mu
\] lo cual implica que el estimador, en promedio, devolvería el
parámetro que nos interesa (este también es \emph{insesgado}).
\end{Ejemplo}

  \bibliography{book.bib,packages.bib}

\end{document}
